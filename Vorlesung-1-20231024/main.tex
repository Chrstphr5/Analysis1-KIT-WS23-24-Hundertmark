%---------------
%---------------
%Danke fürs mitarbeiten. Sorry für den nicht ganz so schönen code!
%---------------
%---------------

\documentclass{article}

\usepackage{ocencfd}

\title{Skript Analysis 1 Vorlesung 1} % Sets article title
\author{\textit{Alle Angaben ohne Gewähr}} % Sets authors name
\authorID{senglert} %Link to your profile ID.
\documentID{Ana1Vorlesungsskript} %Should be alphanumeric identifier
\fileInclude{} %Names of file to attach
\usepackage[ngerman]{babel}
\date{\today} % Sets date for publication as date compiled
\setcounter{section}{-1}

% The preamble ends with the command \begin{document}
\begin{document} % All begin commands must be paired with an end command somewhere

	\maketitle % creates title using information in preamble (title, author, date)
    \section{Beispiele}
        Die Natürlichen Zahlen $\mathbb{N}$ sind alle Zählzahlen ($1,2,3,4,5,...$) \\
         $$n\in \mathbb{N} \text{ ist gerade mit } n=2k\\$$
         $$n\in \mathbb{N} \text{ ist ungerade mit } n=2k-1$$
        \subsection{Behauptung 1}
            Für $n\in \mathbb{N}$ ist $n$ gerade, folgt aus $n^2$ gerade\\\\
            \textbf{Direkter Beweis:}
            \begin{align*}
                n &= 2k \\
                \implies n^2 &= 4k^2 \\
                \, &= 2 \cdot \underbrace{(2k^2)}_{\in \mathbb{N}} \\
                &\implies n^2 \text{ ist gerade}
            \end{align*}
	    \hfill $\Box$
        \subsection{Behauptung 2}
            aus $n^2$ gerade folgt $n$ gerade. Diese Aussage zu treffen ist schwierig, da man mit Wurzeln hantieren muss. Was allerdings einfach zu beweisen ist, ist Behauptung 3
        \subsection{Behauptung 3}
            Aus $n$ ungerade folgt $n^2$ ungerade\\\\
            \textbf{Beweis durch Kontraposition:}\\
                Annahme: $n=2k-1$ mit $k\in \mathbb{N}$
                \begin{align*}
                    n^2 &= (2k-1)^2 \\
                    \, &= 4k^2 - 4k + 1 \\
                    \, &= 2(2k^2 - 2k) + 1 \\
                    \, &= 2(2k(k-1) + 1 \\
                    \, &= 2\underbrace{(2k(k-1)+1)}_{\in \mathbb{N}} - 1 \\
                    &\implies n^2 \text{ ist ungerade}
                \end{align*}
		\hfill $\Box$
            \textbf{Frage: Was hat Beh. 2 mit Beh. 3 zu tun?}\\
            p="$n$ ist gerade", q="$n^2$ ist gerade"\\
            Beh. 2: Aus q folgt p\\
            Beh. 3 Aus nicht p folgt nicht q \textbf{(Kontraposition)}
            \textbf{$$A\implies B \Leftrightarrow \lnot B \implies \lnot A$$}
        \subsection{Beispiel 2}
        Beweis durch Widerspruch: $\sqrt{2}$ ist irrational\\\\
        \textbf{Alternativ Beweis}\\
            Annahme: $\sqrt{2}\in \mathbb{Q}$\\
            $$\text{D.h.} \sqrt{2}=\frac{m}{n} \text{mit }m \in \mathbb{Z}, n\in \mathbb{N}$$
            $$\text{D.h. } A:= \{ n\in \mathbb{N} : \exists m \in \mathbb{Z} \text{ mit } \sqrt{2}=\frac{m}{n}\}$$
            $\sqrt{2}$ ist rational genau dann, wenn A  nicht leer ist.\\
            A ist Teilmenge $\mathbb{N}$\\
            ist A nicht leer, so hat A ein kleinstes Element (\textbf{Prinzip des kleinsten Diebes})\\
            D.h. es existiert $n_*\in A$ mit $n \ge n_*$ deshalb in A $\sqrt{2}=\frac{m}{n_*}$\\
            $$m-n_*=\sqrt{2}n_*-n_*=(\sqrt{2}-1)n_*$$
            Es gilt: $1<\sqrt{2}<2$ $\to$ $0<\sqrt{2}-1< 1$\\
            Also folgt $m-n_*$ ist eine ganze Zahl\\
            und $m-n_*=(\sqrt{2}-1)n_*>0$ also ist $m-n_* \in \mathbb{N} \ge 1$\\
            und $m-n_*=(\sqrt{2}-1)n_*< n_*$\\
            Also ist
            $$\sqrt{2}=\frac{m}{n_*} =
            \frac{m(m-n_*)}{n_*(m-n_*)}=
            \frac{m^2-mn_*}{n_*(m-n_*)}$$
            $$=\frac{2n^2-mn_*}{n_*(m-n_*)}$$
            $$=\frac{n_*(2n_*-m)}{n_*(m-n_*)}=
            \frac{2n_*-m}{m-n_*}=
            \frac{\Tilde{m}}{m-n_*}$$
            wobei $\Tilde{m} \in \mathbb{Z}$\\
            D.h. $m-n_* \in A$\\
            \textbf{Widerspruch} zu $n_*$ ist kleinstes Element von A, da $m-n_*<n_*$
	    \hfill $\Box$
        \subsection{Was ist mit $\sqrt{k}$ mit $k\in \mathbb{N}$}
            \textbf{Beweis:}\\
            Angenommen $\sqrt{\textbf{k}}\notin \mathbb{N}$\\
            D.h. $\sqrt{k}=\frac{m}{n}$ mit $m \in \mathbb{Z}, n\in\mathbb{N}$\\
            $$A:=\{ n \in \mathbb{N}: \exists m\in \sqrt{k}=\frac{m}{n}\}$$
            $n_x$: kleinstes Element von A $\to \sqrt{k}=\frac{m}{n_x}$\\\\
            \textbf{Satz 2:} Sei $k\in \mathbb{N}$, dann ist $\sqrt{k}\in \mathbb{N}$ oder $ \sqrt{k} \in \mathbb{R}$!\\
            Mit was solle man $\sqrt{k}=\frac{m}{n_*}$ erweitern, um wieder einen Widerspruch zu erzeugen?\\
            Für $k\ge 5$ dann ist $\sqrt{k}>2$\\
            Funktioniert $m-ln_*$ für geeignetes $l$?
        
            

\end{document}
